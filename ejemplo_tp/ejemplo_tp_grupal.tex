\documentclass[10pt,a4paper]{article}
\input{AEDmacros}
\usepackage{caratula} % Version modificada para usar las macros de algo1 de ~> https://github.com/bcardiff/dc-tex


\titulo{Descripci\'on del tp}
\subtitulo{Subtítulo del tp}

\fecha{\today}

\materia{Materia de la carrera}
\grupo{Grupo 42}

\integrante{Luis Enrrique Roncal Aranda}{001/01}{luis79enrrique@gmail.com} 
\integrante{Apellido, Nombre2}{002/01}{email2@dominio.com}
\integrante{Apellido, Nombre3}{003/01}{email3@dominio.com}
\integrante{Apellido, Nombre4}{004/01}{email4@dominio.com}
% Pongan cuantos integrantes quieran

% Declaramos donde van a estar las figuras
% No es obligatorio, pero suele ser comodo
\graphicspath{{../static/}}

\begin{document}

\maketitle

\section{Ejemplo de sección}
\subsection{Subsección: ambientes comunes de \LaTeX}

\begin{proc}{grandesCiudades}{\In ciudades : \TLista{Ciudad}}{\TLista{Ciudad}}
	\requiere{sonPositivos(ciudades) \land sinRepetidos(ciudades)}
	\asegura{todosPertenecen(res,ciudades)\land \#ciudadesQueSuperan(ciudades) = \longitud{res}}
	\asegura{ todosSuperanLaCantidad(res)\land sinRepetidos(res)}
\end{proc}

\pred{todosPertenecen}{s,t : \TLista{Ciudad}}{\paraTodo[unalinea]{i}{\ent}{ 0\leq i < \longitud{s} \implicaLuego  t[i] \in s}}
\pred{todosSuperanLaCantidad}{s : \TLista{Ciudad}}{\paraTodo[unalinea]{i}{\ent}{ 0\leq i < \longitud{s} \implicaLuego s[i].habitantes > 50000}}
\pred{sonPositivos}{s : \TLista{Ciudad}}{\paraTodo[unalinea]{i}{\ent}{ 0\leq i < \longitud{s} \implicaLuego  s[i].habitantes \geq 0}}




\subsection{Macros de la cátedra para especificar}

\begin{proc}{nombre}{\In paramIn : \nat, \Inout paramInout : \TLista{\ent}}{tipoRes}
	%    \modifica{parametro1, parametro2,..}
	\requiere{expresionBooleana1}
	\asegura{expresionBooleana2}
	\aux{auxiliar1}{parametros}{tipoRes}{expresion}
	\pred{pred1}{parametros}{expresion} 
\end{proc}

\aux{auxiliarSuelto}{parametros}{tipoRes}{expresion}
% \paraTodo{variable}{tipo}{expresion}
% \existe{variable}{tipo}{expresion}
% Pueden tener [unalinea] para que no se divida en varias lineas
\pred{predSuelto}{parametros}{\paraTodo[unalinea]{variable}{tipo}{algo \implicaLuego expresion}}
\pred{predSuelto}{parametros}{\existe[unalinea]{variable}{tipo}{algo \yLuego expresion}}

\begin{equation}
	\sum\limits_{i=0}^{n} i
	\label{eq:1}
\end{equation}

\end{document}
